\begin{chineseabstract}
随着我国信息化建设的不断推进,以及移动互联网的不断发展,
互联网中的信息开始爆炸式的增长,进入了以短文本信息为主的碎片化信息时代。
如何从浩如烟海的信息碎片中准确的获取有用信息,成了学者及商业公司关注的焦点。
本文主要研究了中文短文本分类的相关技术,改进了传统词向量与字向量的训练模型,
然后以此对经典文本分类模型进行有效改进,
设计并实现了一个适合中文短文本的分类模型。本文的主要工作包括了以下几个方面:
\begin{enumerate}
    \item 通过研究传统词向量模型在中文语料上的不足,本文改进了原有的训练模型,引入
    了中文汉字信息以及偏旁部首信息,让词向量模型能够获得额外的共现信息,
    更加适合中文文本。并通过部首转换机制,
    将部首替换为与其对应的汉字,使词向量模型能够更好地识别具有语义联系的词语
    ,让具有相似语义的单词能够在向量空间中彼此靠近,性能更优,解释性更强。并且以此为基础提出了
    另一个字向量模型,弥补了分词错误对词向量的影响,为后续的分类模型提供了更加丰富的语义信息。
    \item 本文将卷积神经网络与循环神经网组合使用,引入Attention Model技术,
    设计了一个全新的特征提取网络。
    该网络采用了k-max池化与双向循环神经网络技术,具有更好的特征提取能力,能够有效的识别并提取
    文本数据中的语义特征。通过Attention Model技术,
    网络更加专注分类特征的提取,剔除了无效特征,整体上提升了特征向量质量,
    从而提升分类模型的分类效果。
    \item 本文采用双通道的短文本分类模型结构,结合了字向量与词向量的文本数据,
    从两个不同的文本表示中对同一段文本提取文本特征,
    极大的丰富了输入短文本的文本信息,有效提升了分类效果,
    并通过与其他同类模型的对比实验,证明了模型的可行性。
\end{enumerate}


\chinesekeyword{短文本,文本分类,深度学习,Attention技术,词向量}
\end{chineseabstract}



\begin{englishabstract}
With aaa

\englishkeyword{time-domain electromagnetic scattering, time-domain integral equation (TDIE), marching-on in-time (MOT) scheme, late-time instability, plane wave time-domain (PWTD) algorithm}
\end{englishabstract}
    