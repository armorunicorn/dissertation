\chapter{全文总结和展望}
\section{全文总结}
随着我国互联网市场的不断扩大,
网络上充斥了大量有价值或无价值的文本信息,同时由于移动互联设备的逐渐普及,
这些文本信息半数以上以短文本的形式存在。文本分类技术作为自然语言处理的基础技术,
能够有效的理解和梳理这些短文本信息,为后续的文本挖掘工作提供坚实的基础,在
学术研究与商业应用领域都有广泛的使用,具备广阔的应用前景。

本文主要研究了深度学习在中文短文本分类方面的应用,通过研究汉字字词和部首语义上的联系,
以及改进字/词向量模型和特征提取网络,实现了一个新型的文本分类网络模型,
,并结合实际场景,构建了一个短文本分类
服务系统。本文的主要贡献如下:
\begin{enumerate}
    \item 对文本分类技术的发展进行了大致介绍,简述了传统文本分类技术的发展现状,
    并对深度学习下的自然语言处理技术进行了简要描述。
    \item 分析了中文词语和词语中包含汉字及相关部首在语义上的内在联系,在CBOW词向量模型的基础
    上引入了汉字及部首信息,构造了一种新的词向量训练模型,使得训练出的词向量更加切合中文文本,
    具有更强的语义相关性。
    \item 在词向量模型的基础上提出了另一个字向量模型,利用汉字部首的语义信息训练中文字向量,
    让语义上有联系的汉字能够在向量空间中更加靠近,补充了词向量模型的不足。
    \item 结合卷积神经网络与循环神经网络,通过k-max池化与双向循环神经网络技术,
    设计了一个新的特征提取网络。并引入Attention Model技术,让网络能够专注分类特征的提取,
    整体上提升了特征向量质量。
    \item 结合字向量与词向量的文本数据,设计了双通道的短文本分类模型,从两个不同的文本表示中同时提取文本特征,
    极大的丰富了输入短文本的文本信息,有效提升了分类结果,
    并通过与其他同类模型的对比实验,证明了模型的可行性。
    \item 综合本文设计的词/字向量模型与分类网络,结合网络爬虫与WEB服务等模块,实现了一个
    基于网络数据的中文短文本分类服务系统,为外部用户提供持续可靠的短文本分类服务。
\end{enumerate}
\section{后续工作展望}
本文的重点在于寻找一个适合中文短文本的分类模型,由于时间有限,其中仍然有一些改进空间:
\begin{enumerate}
    \item 卷积神经网络与循环神经网络近几年在深度学习领域都有很大发展,可以考虑
    使用其他的网络结构模型改进本文设计的卷积循环特征提取网络,
    如用GRU(Gated Recurrent Unit,门控循环单元)替换LSTM。
    \item 可以使用其他Attention Model的实现方式来筛选文本特征向量,
    通过更复杂的方式获得特征权重,从而进一步优化特征向量。
    \item 本文最后实现的中文短文本分类服务系统,虽然测试有效,但还有很多地方可以
    改进,比如增加更复杂的爬虫管理模块、提升系统计算效率、增加用户管理功能模块等,
    逐步提升系统性能,进一步完善系统功能。
\end{enumerate}