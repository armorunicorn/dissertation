\thesischapterexordium
\section{研究工作的背景与意义}
随着信息化进程的不断推进,我国互联网市场空前繁荣。从WEB2.0时代开始,到如今的“互联网+”时代,互联网已经融入了人们的生活中,也对各行各业产生了深远的影响。
根据中国互联网络信息中心发布的《第40次中国互联网络发展状况统计报告\footnote{\url{http://www.cnnic.net.cn/hlwfzyj/hlwxzbg/hlwtjbg/201708/P020170807351923262153.pdf}}》(以下简称《报告》)显示,
截止到2017年6月,我国网民规模达到7.51亿,其中手机网民更是达到7.24亿,占总体网民的96.3\%。
同时,使用率排名前三的互联网应用分别是即时通信(92.1\%)、网络新闻(83.1\%)、搜索引擎(81.1\%)。
可以明显看出,人们使用互联网以及获取信息的方式,已经从以前的桌面台式电脑,转变为移动端的手机、IPAD等掌上设备。
移动设备的盛行,使得人们发布和接收信息都更加方便,据统计,在2012年,微博用户已经增长到3.68亿,其中69\%通过移动设备登陆,每天能产生1.17亿条微博。

\section{国内外研究现状}
\section{课题主要内容}
\section{本论文的结构安排}
